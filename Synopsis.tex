
A double-weighted power mean model for the excess Gibbs energy of fluid mixtures, combined with the cubic equations of state, is applied to liquid-liquid phase equilibria. Algorithms for binary model parameter estimation and phase diagram construction were formulated and implemented using the Python programming language. The suggested method was applied successfully to correlate experimental liquid-liquid equilibrium data of a number of binary and ternary mixtures using the double-weighted power mean, NRTL and UNIQUAC models.\\

The following conclusions were drawn:\
\begin{itemize}
\item The double-weighted power mean model, in the form of the modified Wilson model, has been found to be adequately flexible to reproduce experimental binary and ternary data with single two-phase regions.\
\item The pseudo-analytical approach, formulated for excess Gibbs energy model parameter estimation, was applied successfully to all binary mixtures studied and ternary mixtures with single two-phase regions.\
\item Similarly, the pseudo-analytical approach for equilibrium phase calculation, was applied successfully to all binary mixtures studied and ternary mixtures with single two-phase regions.\
\item The pseudo-analytical approach was easily implemented and produced repeatable results.\
\end{itemize}

The following recommendations were made:\
\begin{itemize}
\item A modified pseudo-analytical approach should be applied to fit the adjustable parameters of the double-weighted power mean model, as well as the binary interaction parameters, to data of systems containing multiple liquid miscibility gaps.\
\item By correlating unusual liquid-liquid equilibrium data, the ability of the double-weighted power mean model to accurately reproduce such behaviour can be investigated.\
\item A statistical method for tie-line selection or overall parameter optimization, in conjunction with the pseudo-analytical method for parameter estimation from ternary data, should be investigated.\
\item The use of a suitable data reduction technique, for binary and ternary data, should be explored in order to increase the confidence in the calculated model parameters.
\end{itemize}

