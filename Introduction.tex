                                                                                                                                           
Equilibrium based separation processes play a significant role in the chemical processing industry. The accurate knowledge of phase behaviour is essential for the design and operation of such units. It is often the case that, especially for multicomponent mixtures, experimental data at the applicable temperatures and pressures are not readily available. Procedures for obtaining experimental data are often time consuming, costly and tedious. Such expenditure in time and money is usually not warranted in the initial stages of process development. Even in later stages of process design and optimization, and when experimental data is readily available, methods to correlate these data and predict phase behaviour are required.\\

Excess Gibbs energy models, or activity coefficient models, are commonly used to model liquid-liquid equilibrium for the design and optimization of extraction equipment. Due to the close proximity of molecules, resulting in more frequent molecular interaction, non-ideal behaviour in the liquid phase is more pronounced than in the vapour phase. The applicability of a specific excess Gibbs energy model is determined by it's ability to accommodate these variations from the ideal.\\

In addition, in order for these models to be of any practical use, the adjustable model parameters need to be known. A large number and variety of components are normally present in industrial process streams. The binary interaction parameters of Gibbs energy models are consequently determined by fitting these models to binary and ternary mixture data, and then used in a predictive manner to model industrial multi-component mixtures. With modern day computing power and advanced numerical methods, complex liquid phase behaviour can be correlated and modelled. However, parameter estimation can be challenging and even equilibrium calculations, when model parameters are known, is a complex mathematical problem.\\ 

A double-weighted power mean mixture model for the excess Gibbs energy of fluid mixtures, combined with the cubic equations of state, has been proposed by \citeauthor{WeightedPowerMeanModel} in \citeyear{WeightedPowerMeanModel}. An investigation was launched into the ability of the double-weighted power mean model to accommodate liquid-liquid phase equilibria.\\

A method was formulated whereby the double-weighted power mean model could be correlated to experimental binary and ternary liquid-liquid equilibrium data. Experimental data for a number of binary and ternary mixtures were obtained and the parameter estimation algorithm was implemented with the Python programming language. The software was also used to determine the model parameters for the NRTL and UNIQUAC models from the same binary solubility data, in order to validate the algorithm and compare the performance of the double-weighted power mean model. Methods for constructing the binary and ternary phase diagrams, predicted by the calculated model parameters, were also developed and implemented using Python.\\

In the current investigation only mixtures of organic compounds and water, obtained from the Dechema liquid-liquid equilibrium data collection, were studied. For all these mixtures, excluding 1-Hexanol, Nitro-Methane and Water, only one physical miscibility gap is observed. The Dechema collection contains experimental data for ternary mixtures with up to 3 phases in equilibrium. While data therefore exists for systems with more complex phase behaviour, it is rarely encountered and beyond the scope of this investigation.\\

